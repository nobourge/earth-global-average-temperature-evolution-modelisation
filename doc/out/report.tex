\documentclass[12pt]{article}
\usepackage[utf8]{inputenc}
\usepackage{graphicx}
\usepackage{amsmath}
\usepackage{hyperref}

\title{Modeling and Simulation of Earth's Average Global Temperature}
\author{Palmisano Luca \\ Bourgeois Noé}
\date{November 2023}

\begin{document}

\maketitle

\begin{abstract}
\noindent
The present report outlines the findings from a series of simulations designed to model the Earth's average global temperature. Utilizing a range of models, from a basic Energy Balance Model (EBM) to more complex variations incorporating factors like albedo and emissivity, this study aims to provide insights into the dynamics of Earth's climate system and the factors influencing its temperature.
\end{abstract}

\section{Introduction}
Climate change, characterized by alterations in Earth's average temperature, is a critical global issue. This report focuses on modeling the Earth's temperature using various simulation models. The objective is to understand how different parameters affect global temperature. We explore several models, including the Basic EBM and its extensions, to simulate and analyze temperature changes under different conditions.

\section{Methodology}
The methodology employed in this project involves simulating Earth's temperature using several models. The Basic EBM is the starting point, which is then expanded to include factors like albedo variation and emissivity. The simulations are conducted using the Octave software, leveraging its numerical computation capabilities to solve complex differential equations inherent in the models.

\section{Results}
% Fill in your simulation results here.
% Use graphs and tables to present your data.
% Example of including a figure:

\begin{figure}[ht]
\centering
% \includegraphics[width=0.8\textwidth]{path_to_your_figure.png}
\caption{Simulation results for the Basic EBM model.}
\label{fig:basicEBM}
\end{figure}

% Describe the results you obtained from each model.
% Discuss the trends, patterns, and any anomalies observed.

\section{Discussion}
% Analyze the results you've presented.
% How do the models compare with each other?
% Discuss the implications of your findings, especially in the context of climate change.
% Are there any limitations or notable aspects of the models that affected the results?

\section{Conclusion}
% Summarize your main findings.
% Reflect on what these results mean for our understanding of Earth's climate system.

\end{document}
